\documentclass[10pt, letterpaper]{article}

% Packages:
\usepackage[
    ignoreheadfoot, % set margins without considering header and footer
    top=2 cm, % seperation between body and page edge from the top
    bottom=2 cm, % seperation between body and page edge from the bottom
    left=2 cm, % seperation between body and page edge from the left
    right=2 cm, % seperation between body and page edge from the right
    footskip=1.0 cm, % seperation between body and footer
    % showframe % for debugging 
]{geometry} % for adjusting page geometry
\usepackage{titlesec} % for customizing section titles
\usepackage{tabularx} % for making tables with fixed width columns
\usepackage{array} % tabularx requires this
\usepackage[dvipsnames]{xcolor} % for coloring text
\definecolor{primaryColor}{RGB}{0, 0, 0} % define primary color
\usepackage{enumitem} % for customizing lists
\usepackage{fontawesome5} % for using icons
\usepackage{amsmath} % for math
\usepackage[
    pdftitle={Zhuhan Bao's CV},
    pdfauthor={Zhuhan Bao},
    pdfcreator={LaTeX with RenderCV},
    colorlinks=true,
    urlcolor=primaryColor
]{hyperref} % for links, metadata and bookmarks
\usepackage[pscoord]{eso-pic} % for floating text on the page
\usepackage{calc} % for calculating lengths
\usepackage{bookmark} % for bookmarks
\usepackage{lastpage} % for getting the total number of pages
\usepackage{changepage} % for one column entries (adjustwidth environment)
\usepackage{paracol} % for two and three column entries
\usepackage{ifthen} % for conditional statements
\usepackage{needspace} % for avoiding page brake right after the section title
\usepackage{iftex} % check if engine is pdflatex, xetex or luatex

% Ensure that generate pdf is machine readable/ATS parsable:
\ifPDFTeX
    \input{glyphtounicode}
    \pdfgentounicode=1
    \usepackage[T1]{fontenc}
    \usepackage[utf8]{inputenc}
    \usepackage{lmodern}
\fi

\usepackage{charter}

% Some settings:
\renewcommand{\baselinestretch}{1.12}
\raggedright
\AtBeginEnvironment{adjustwidth}{\partopsep0pt}
\pagestyle{empty}
\setcounter{secnumdepth}{0}
\setlength{\parindent}{0pt}
\setlength{\topskip}{0pt}
\setlength{\columnsep}{0.25cm}
\pagenumbering{gobble}

% 调整正文字体大小
\renewcommand{\normalsize}{\fontsize{12}{14.5}\selectfont}
\renewcommand{\small}{\fontsize{11}{13}\selectfont}

% 调整页边距
\geometry{
    top=2.5cm,
    bottom=2.5cm,
    left=2.2cm,
    right=2.2cm
}

% section标题与正文间距加大
\titlespacing*{\section}{-1pt}{0.6cm}{0.45cm}

% 横向划分线:section标题下自动加横线
\titleformat{\section}{\needspace{4\baselineskip}\bfseries\fontsize{20}{22}\selectfont}{}{0pt}{}[\vspace{1pt}\titlerule]

% 项目之间增加间距
\newcommand{\entryvspace}{\vspace{0.22cm}}

\titlespacing{\section}{
    % left space:
    -1pt
}{
    % top space:
    0.35 cm
}{
    % bottom space:
    0.25 cm
} % section title spacing

\renewcommand\labelitemi{$\vcenter{\hbox{\small$\bullet$}}$} % custom bullet points
\newenvironment{highlights}{
    \begin{itemize}[
        topsep=0.10 cm,
        parsep=0.10 cm,
        partopsep=0pt,
        itemsep=0pt,
        leftmargin=0 cm + 10pt
    ]
}{
    \end{itemize}
} % new environment for highlights


\newenvironment{highlightsforbulletentries}{
    \begin{itemize}[
        topsep=0.10 cm,
        parsep=0.10 cm,
        partopsep=0pt,
        itemsep=0pt,
        leftmargin=10pt
    ]
}{
    \end{itemize}
} % new environment for highlights for bullet entries

\newenvironment{onecolentry}{
    \begin{adjustwidth}{
        0 cm + 0.00001 cm
    }{
        0 cm + 0.00001 cm
    }
}{
    \end{adjustwidth}
} % new environment for one column entries

\newenvironment{twocolentry}[2][]{
    \onecolentry
    \def\secondColumn{#2}
    \setcolumnwidth{\fill, 4.5 cm}
    \begin{paracol}{2}
}{
    \switchcolumn \raggedleft \secondColumn
    \end{paracol}
    \endonecolentry
} % new environment for two column entries

\newenvironment{threecolentry}[3][]{
    \onecolentry
    \def\thirdColumn{#3}
    \setcolumnwidth{, \fill, 4.5 cm}
    \begin{paracol}{3}
    {\raggedright #2} \switchcolumn
}{
    \switchcolumn \raggedleft \thirdColumn
    \end{paracol}
    \endonecolentry
} % new environment for three column entries

\newenvironment{header}{
    \setlength{\topsep}{0pt}\par\kern\topsep\centering\linespread{1.5}
}{
    \par\kern\topsep
} % new environment for the header

\newcommand{\placelastupdatedtext}{% \placetextbox{<horizontal pos>}{<vertical pos>}{<stuff>}
  \AddToShipoutPictureFG*{% Add <stuff> to current page foreground
    \put(
        \LenToUnit{\paperwidth-2 cm-0 cm+0.05cm},
        \LenToUnit{\paperheight-1.0 cm}
    ){\vtop{{\null}\makebox[0pt][c]{
        \small\color{gray}\textit{Last updated in September 2024}\hspace{\widthof{Last updated in September 2024}}
    }}}%
  }%
}%

% save the original href command in a new command:
\let\hrefWithoutArrow\href

% new command for external links:


\begin{document}
    \newcommand{\AND}{\unskip
        \cleaders\copy\ANDbox\hskip\wd\ANDbox
        \ignorespaces
    }
    \newsavebox\ANDbox
    \sbox\ANDbox{$|$}

    \begin{header}
        \fontsize{25 pt}{25 pt}\selectfont Zhuhan Bao

        \vspace{5 pt}

        \normalsize
        \mbox{Zhejiang, China}%
        \kern 5.0 pt%
        \AND%
        \kern 5.0 pt%
        \mbox{\hrefWithoutArrow{mailto:hengyuhan3762@gmail.com}{hengyuhan3762@gmail.com}}%
        \kern 5.0 pt%
        \AND%
        \kern 5.0 pt%
        \mbox{\hrefWithoutArrow{https://github.com/BaoZhuhan}{github.com/BaoZhuhan}}%
    \end{header}

    % ===== 教育经历 =====
    \section{Education}
    \begin{twocolentry}{Sep 2023 -- Jul 2027}
        \textbf{Hangzhou City University} B.Eng. in Software Engineering\\
    \end{twocolentry}
    

    % ===== 研究经历 =====
    \section{Research Experience}
    \begin{onecolentry}
        \textbf{Visiting Student}, Advisor: Prof. Peiyu Liu \hfill May 2025 -- Sep 2025\\
         \href{https://nesa.zju.edu.cn/}{Zhejiang University NESA Research Lab}\\\textbf{Zhejiang University}, Hangzhou Zhejiang China\\
         \vspace{0.03 cm}
        \textbf{Systematic Security Impacts Investigation of MCP Servers in AI Code Editors}
    \end{onecolentry}
    \begin{onecolentry}
        This project systematically investigates the security impacts of Model Context Protocol (MCP) servers in AI code editors. My responsibilities include evaluating MCP server support for the SDLC, developing automated workflows for their deployment, and analyzing potential security risks in code editor integration.
    \end{onecolentry}
    \entryvspace
    \begin{onecolentry}
        \textbf{Project Leader}, Advisor: \href{https://scholar.google.com/citations?hl=zh-CN&user=48cqMXkAAAAJ}{Prof. Lin Sun} \hfill Mar 2025 -- Mar 2026\\
        The School of Computer and Computing Science\\ \textbf{Hangzhou City University}, Hangzhou Zhejiang China\\
        \vspace{0.03 cm}
        \textbf{Medical Multimodal Dataset Construction and Large Model Research Based on Online Teaching Videos}
    \end{onecolentry}
    \begin{onecolentry}
        This project aims to construct a high-quality bilingual (Chinese-English) pathology image-text dataset from online educational videos and fine-tune the CLIP model using the Lora method to enhance the performance and accuracy of intelligent pathology diagnosis and medical knowledge retrieval. As project leader, I am mainly responsible for developing web crawlers, aligning the dataset, and fine-tuning the model.
    \end{onecolentry}

    % ===== 项目经历 =====
    \section{Project Experience}
    \begin{twocolentry}{Jan 2025 -- Feb 2025}
        \textbf{RNA 5-Methylcytosine Construction Optimization}\\
        \textcolor{gray}{2025 ASC Student Supercomputer Challenge Preliminary}
    \end{twocolentry}
    \begin{onecolentry}
        This project focuses on optimizing the detection workflow for RNA 5-methylcytosine (m5C) modification sites, aiming to enhance both precision and computational efficiency. I was responsible for designing and implementing automated scripts for data cleaning, alignment, deduplication, and statistical filtering, as well as tuning parameters to achieve high accuracy and reduced runtime. The project achieved significant improvements in automation, runtime, and memory usage, and established a robust, reproducible pipeline for m5C site detection.
    \end{onecolentry}
    \entryvspace
    \begin{twocolentry}{Jan 2025 -- Feb 2025}
        \textbf{AlphaFold3 Inference Optimization}\\
        \textcolor{gray}{2025 ASC Student Supercomputer Challenge Preliminary}
    \end{twocolentry}
    \begin{onecolentry}
        This project focused on optimizing the inference workflow of AlphaFold3, a state-of-the-art protein structure prediction model. My responsibilities included building the Docker-based environment, profiling and analyzing computational hotspots (especially in the diffusion model), and applying JIT (Just-In-Time) compilation and custom operator optimizations for both GPU and CPU inference. Through targeted improvements such as JAX JIT acceleration and flash attention adjustments, the project achieved significant reductions in inference time on multi-GPU and CPU platforms, while maintaining prediction accuracy and reproducibility. The work also involved systematic benchmarking and analysis of hardware utilization and optimization effects.
    \end{onecolentry}
    \entryvspace
    \begin{twocolentry}{Jul 2024 -- Jul 2024}
        \textbf{CPUBench Optimization}\\
        \textcolor{gray}{PAC National Parallel Application Challenge Preliminary}
    \end{twocolentry}
    \begin{onecolentry}
        This project focused on optimizing the performance of the CPUBench benchmark. My responsibilities included tuning compilation parameters, optimizing math libraries, and improving the build environment to maximize computational efficiency. The project achieved notable improvements in execution speed, resource utilization, and reproducibility of the benchmarking workflow.
    \end{onecolentry}

    % ===== 学习经历 =====
    \section{Learning Experience}
    \begin{onecolentry}
        \textbf{Competitions:}
    \end{onecolentry}
    \begin{twocolentry}{\textbf{Second Prize}}
        16th Lanqiao Cup C/C++ Programming, Zhejiang Province
    \end{twocolentry}
    \vspace{0.05 cm}
    \begin{twocolentry}{\textbf{Third Prize}}
        15th Lanqiao Cup C/C++ Programming, Zhejiang Province
    \end{twocolentry}
    \vspace{0.05 cm}
    \begin{twocolentry}{\textbf{Second Prize}}
        2024 National Parallel Application Challenge (PAC)
    \end{twocolentry}
    \vspace{0.05 cm}
    \begin{twocolentry}{\textbf{Second Prize}}
        2025 ASC Student Supercomputer Challenge Preliminary
    \end{twocolentry}
    \vspace{0.05 cm}
    \begin{onecolentry}
        \textbf{Honors:}
    \end{onecolentry}
    \vspace{0.05 cm}
    \begin{twocolentry}{\textbf{2024}}
        Academic Excellence Scholarship
    \end{twocolentry}
    \vspace{0.05 cm}
    \begin{twocolentry}{\textbf{2024}}
        Outstanding Student
    \end{twocolentry}
    \vspace{0.05 cm}
    \begin{twocolentry}{\textbf{2024}}
        Excellent Student Leader
    \end{twocolentry}

    % ===== 技能与证书 =====
    \section{Skills}
    \begin{onecolentry}
        \textbf{Certificates:}~CET-4, CET-6, PRC Driver's License (C1) \\
        \vspace{0.08cm}
        \textbf{Programming Languages:}~C/C++, Python, Java, Bash, SQL \\
        \vspace{0.08cm}
        \textbf{Skills:}~Parallel Computing, High Performance Computing, Machine Learning, Data Analysis, Algorithm Design \\
        \vspace{0.08cm}
        \textbf{Software \& Platforms:}~Linux, Vtuner, Perf, SSH, Docker, Git, Conda, PyTorch
    \end{onecolentry}
\end{document}